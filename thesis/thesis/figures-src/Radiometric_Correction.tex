\tikzset{external/export next=false}

\tikzset{
	basic/.style  = {draw, text width=2cm, drop shadow, font=\sffamily, rectangle},
	root/.style   = {basic, rounded corners=2pt, thin, align=center,
		fill=green!30},
	level 2/.style = {basic, rounded corners=6pt, thin, align=center, fill=green!60,
		text width=6.5em},
	level 3/.style = {basic, thin, align=left, fill=pink!60, text width=9em}
}

\begin{tikzpicture}[
level 1/.style={sibling distance=50mm},
edge from parent/.style={->,draw},
>=latex]

% root of the the initial tree, level 1
\node[root] {Radiometric Correction}
% The first level, as children of the initial tree
child {node[level 2] (c1) {Sensor Calibration}}
child {node[level 2] (c2) {Sun Angle /Surface Slope}}
child {node[level 2] (c3) {Atmospheric Correction}};

% The second level, relatively positioned nodes
\begin{scope}[every node/.style={level 3}]
\node [below of = c1, xshift=30pt] (c11) {Sensitivity of Detectors};
\node [below of = c11] (c12) {Vignetting Effecting};

\node [below of = c2, xshift=25pt] (c21) {Sun Angle Effect};
\node [below of = c21] (c22) {Topographic Effect};

\node [below of = c3, xshift=25pt] (c31) {Topographic Effect};
\node [below of = c31] (c32) {Scattering};
\end{scope}

% lines from each level 1 node to every one of its "children"
\foreach \value in {1,2}
\draw[->] (c1.195) |- (c1\value.west);

\foreach \value in {1,2}
\draw[->] (c2.195) |- (c2\value.west);

\foreach \value in {1,2}
\draw[->] (c3.195) |- (c3\value.west);
\end{tikzpicture}

	