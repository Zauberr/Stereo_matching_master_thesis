\addchap*{Abstract}

Image registration is an important and fundamental topic in computer version, which aim to find the corresponding pixel or displacement field between two consecutive images within an image sequence. The purpose of this thesis is to derive a new algorithm to improve the image registration of piecewise planar world by estimating the multiple homography of planar scenes in consecutive images or stereo images. In general, the homography matrix can only be used to register images with one large flat plane. But the proposed algorithm expands the single homography matrix to multiple homography matrices, which is linked by constant variables. It realizes the registration of planar sub-patches in the image. In addition, the estimated multiple homography contains the information of Fundamental matrix, which can be used to follow the camera path. Specifically, a mathematical model of the algorithm is deduced through the existing research results and related theoretical basis. Multiple homographies for different plane patches are used to build the mapping model, after initialization, the parameters of model are iteratively refined in the way of Gauss-Newton algorithm until catching the stopping criteria. At the same time, a new dataset is also built in order to get a more comprehensive evaluation of the algorithm in the thesis. Finally, the algorithm is implemented and applied to both public datasets and self-built dataset to do the evaluation. The self-built dataset is uploaded at : \url{https://github.com/Zauberr/Multi-H-Dataset}.