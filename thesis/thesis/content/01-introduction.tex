\chapter{Introduction}
\section{Motivation}
The master thesis aims to invent a new algorithm to improve image registration for a piecewise planar world. The algorithm estimate compatible homographies between multiple plane patches in an image pair. The image registration technology is used to do the estimation. The result of algorithm can be stated as the epipolar geometry or a set of multiple homography matrices, which can be also used to derive the Fundamental matrix.  For future work, it will be used for following the camera path, camera calibration and 3D reconstruction.

This algorithm is related to image registration. It is well suited for image matching which is a key component in almost all image analysis processes. It is essential for a wide range of application, such as navigation, guidance, automatic surveillance, robot vision, and graphics science. 

This thesis was prepared in association with department VID of Fraunhofer IOSB that deals with the automatic evaluation of signals from imaging sensors. A special focus is on the evaluation of video data on a moving platform. Such sensor system is used, for example, in the field of reconnaissance and surveillance as an integrated component in flying and space-based platform. Therefore, the algorithm is designed to deal especially with aerial video images. Corresponding to this, the algorithm is applied to  consecutive images, in which there are large planar objects.

Basically,  the objects in aerial video are flat earth surface.  This is in line with the applicable conditions of homography. Global homography transformation can be used to warp the big flat dominated plane in the consecutive images. But in most case, there are not only one dominated plane in image or planes in  image are not flat enough, like terraces and mountains. In order to track them, more than one homography between an image pair are used. They can't be chosen freely, but have to obey some rules. The proposed algorithm is to find these compatible homographies between image pairs and use it to do the image registration of sub-patches and refine the epipolar geometry or fundamental matrix. Further, the fundamental matrix is important for searching and detecting the correspondences in images. 

\section{Related Work}
Wojciech Chojnacki, Zygmunt L. Szpak, Michael J.Brooks and Anton van den Hengel have sorted out the structure of multiple homography and present an approach for estimation a set of interdependent homography matrices linked together by latent variables \cite{chojnackiEnforcingConsistencyConstraints2015} \cite{chojnackiMultipleHomographyEstimation2010}. The algorithm proposed in this thesis uses the same structure of multiple homography.  The difference is that the proposed algorithm uses image registration technology to estimate the multiple homography. A W Gruen has applied the adaptive least squares correlation to image matching, it allows for simultaneous radiometric corrections and local geometrical image shaping, whereby the system parameters are automatically assessed, corrected, and thus optimized during the least squares iterations \cite{gruenAdaptiveLeastSquares1985a}. The proposed algorithm also based on the least square correlation, but the result of  estimation are multiple homography matrices between different planar patches in the scene i.e. the proposed algorithm combines the least square iteration and homography transformation to do the image registration. The radiometric corrections introduced in \cite{gruenAdaptiveLeastSquares1985a} is also used in the proposed algorithm to improve the accuracy of the result.

\section{Goal of This Thesis}
The goal of this thesis is to derive a new algorithm to improve image registration for a piecewise planar world with estimation of multiple homography. The main goal of the proposed algorithm is to estimate compatible multiple homography matrices of different sub-patches in the image pair. The result of the proposed algorithm can be used to refine the epipolar geometry or fundamental matrix. And it can also be used to register the sub-patches in image pair with acceptable error. 

In order to achieve this goal, the thesis contains several consecutive goals. First of all, the derivation of the mathematical basic of the algorithm is completed, and secondly the implementation of the algorithm in a program is provided. Finally, a reasonable estimation of multiple homography matrices between image pair is got through the realization of the proposed algorithm. Moreover, an own test scenario is provided to collect test images for the algorithm in a virtual environment. The estimation performance of the algorithm is evaluated based on some public datasets and the self-built dataset.

\section{Organization}
This thesis is structured as follows: \cref{ch:Background} presents some background about image registration, homography and lest square correlation for the thesis. The main idea of the proposed algorithm is described in \cref{sec:main idea}. The derivation of the algorithm and implementation for image registration and parameter estimation process are shown in \cref{sec:Implementation}. Here some considerations for the program of algorithm is introduced in \cref{sec:Program}. Then the result of algorithm is evaluated in \cref{ch:Evaluation}. Finally, we conclude this thesis in \cref{ch:Conclusion}.

